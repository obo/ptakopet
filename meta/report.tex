% !TeX spellcheck = cs_CZ

\documentclass[a4paper]{article}
\usepackage[english]{babel}
\usepackage[utf8x]{inputenc}
\usepackage[T1]{fontenc}
\usepackage{listings}
\usepackage[a4paper,margin=2cm]{geometry}
\usepackage{amsmath}
\usepackage{graphicx}
\usepackage[colorinlistoftodos]{todonotes}
\usepackage[colorlinks=true, allcolors=blue]{hyperref}
\usepackage{wasysym} % smileys
\usepackage{fancyhdr}
\setlength\parindent{0pt} % indent

% my commands:
\newcommand{\n}{\newline}
\newcommand{\tab}{\hspace{1cm}}

\begin{document}

\title{Ptakopět \\ \large Překlad tam a kntrolně zpět}
\author{Vilém Zouhar}
\date{2018 November}
\maketitle 

% https://www.cs.uic.edu/~jbell/CourseNotes/OO_SoftwareEngineering/SE_Project_Report_Template.pdf

\section*{Project}
\subsection*{Overview}
Ptakopět (Překlad Tam A KOntrolně zPĚT) is an agile browser agnostic plugin, which aims to help peoplet translate to languages, where they are unable to verify the quality of the translation, by offering backwards translation.

\subsection*{The purpose of the project}
The issue at hand is known as \textit{Outbound Translation}. Many users use machine translation to verify their own translation, or can at least affirm, that the machine translation is valid. Translation systems are, however, not perfect and despite their great power, they tend to perform poorly on unprecedenced phenomena (unknown words, unexpected word order, etc.). Users translating to languages, which they don't master enough to validate the translation could use Ptakopět to verify (and assure them), that the machine translation output is valid. As many encounters with foreign languages happen on the Internet, Ptakopět is developed as a browser extension.

\section*{Other solutions}
Backwards translation is only one of the solutions to \textit{Outbound Translation}

\subsection*{The scope}


\section*{Implementation}

\section*{Evaluation}

\section*{Possible extensions}


\section*{Conclusion}

\end{document}